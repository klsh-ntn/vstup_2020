\noindent{\Huge\scshape М}{\LARGE\scshape едицина}\\
\rule[0.5\baselineskip]{\textwidth}{1pt}

\vspace{0\baselineskip}

\subsection*{Задание 13.}
У пациента низкий уровень кальция в крови — 1,5~ммоль/л (при норме 2,1 – 2,56~ммоль/л). Предположи, с чем может быть связано такое снижение? Где в организме используется кальций и каким образом? Опиши физиологию/биохимию процессов. Как регулируется уровень кальция в крови? Откуда кальций поступает в кровь? Без какого вещества кальций не будет усваиваться и почему?

\subsection*{Задание 14.}
Одним из симптомов сахарного диабета может быть исходящий от пациента запах ацетона. Какое прогностическое значение имеет этот симптом? Опиши патогенетическую цепочку его возникновения. Что такое сахарный диабет и какие его виды выделяют? Чем они различаются? Опиши возможные осложнения заболевания. Опиши механизм действия инсулина и его роль. Какие органы и ткани зависят от инсулина, а какие нет, и почему? 

\subsection*{Задание 15.}
Национальный календарь прививок — это документ, регламентирующий обязательные прививки, которые может получить каждый человек по полису обязательного медицинского страхования, но есть много прививок, которые не входят в этот обязательный перечень. Предложи три прививки, которые, на твой взгляд, стоит внести в календарь. Обоснуй свой выбор, опираясь на эпидемиологическую ситуацию. Почему от некоторых заболеваний, несмотря на их значимость, не существует вакцин? Какие виды вакцин выделяют? Для чего нужна вакцинация? Что такое иммунитет и как он вырабатывается?
