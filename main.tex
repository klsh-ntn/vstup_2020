\documentclass{article}

\usepackage[a5paper, %
		top=1.1truecm, %
		bottom=1.1truecm, %
		right=1.5truecm, %
		left=1.5truecm, %
		twoside, %
		includeheadfoot, %
		headsep=12pt, %
		footskip=18pt %
]{geometry}
\usepackage{setspace}
\usepackage{wrapfig}
\usepackage{euler} % bb: use modern eulervm? 
\usepackage[usenames]{color} % цвета

\usepackage[euler]{textgreek}
%\usepackage[utf8]{inputenc} % подразумевается в XeLaTeX
\usepackage[X2,T2A]{fontenc} % для математических акцентов
\newcommand{\Е}{{\fontencoding{X2}\selectfont\CYRYAT}} % Буква Ять ЗАГЛАВНАЯ на рус. клавише «Ее»
\newcommand{\е}{{\fontencoding{X2}\selectfont\cyryat}} % Буква Ять строчная

%\usepackage[greek,icelandic,russian]{babel} % bb: NO babel with xelatex

\usepackage{fontspec} %% выбор шрифтов
\usepackage{polyglossia} %% bb: instead of babel

\setmainlanguage{russian} %% bb: instead of babel
\setotherlanguages{greek,icelandic} %% bb: instead of babel

\setmainfont[Ligatures=TeX,
			 Path = fonts/,
             BoldFont=MuseoSansCyrl-700.otf,
             ItalicFont=CharterOSC-Italic.otf,
             BoldItalicFont=MuseoSansCyrl-700Italic.otf,
             SmallCapsFont=CharterSCC.otf
             ]{CharterOSC.otf}

%% bb: here we should specify correct font!:
\newfontfamily{\cyrillicfont}{Linux Libertine O}
\newfontfamily{\cyrillicfonttt}{Linux Libertine O}


\usepackage{amsmath, amssymb}
\usepackage[nointegrals]{wasysym} % bb: moved after amsmath according to https://tex.stackexchange.com/questions/336043/


\usepackage{booktabs} % bb: added for beautiful tables
\usepackage{shapepar}
\usepackage{anyfontsize}
\usepackage{listings}
%\usepackage[dvips] {graphicx} % dvips
\usepackage{calc}
\usepackage{rotating}
\usepackage{graphicx}
%\usepackage{pscyr}
\usepackage{picinpar}
\usepackage{hyperref}
\usepackage{multicol}
\setlength{\columnsep}{1cm}
\usepackage{multirow}
\usepackage{epic}
\usepackage{epigraph}
\usepackage{eepic}
\usepackage{paralist}
\usepackage{lscape}
\usepackage{comment}

\usepackage{version}
\usepackage[version = 3]{mhchem}
\usepackage{enumitem} % дополнительные плюшки для списков

\AddEnumerateCounter{\asbuk}{\russian@alph}{щ} % для списков с русскими буквами
\setlist[enumerate, 1]{label=\asbuk*),ref=\asbuk*}

\includeversion{print}
%\excludeversion{print}

%%%%%%%%%%%%%%%%%%%%%%%%%%%%%%%%%%%%%%%%%%%%%%%%%%%%%%%%%%%%%%%%%%%%%%%%%%%%%
%	Низкоуровневая настройка
%%%%%%%%%%%%%%%%%%%%%%%%%%%%%%%%%%%%%%%%%%%%%%%%%%%%%%%%%%%%%%%%%%%%%%%%%%%%%

\input letterspacing.tex

	\parskip=1pt

	\tolerance=350
	\pretolerance=100
	\widowpenalty=500
	\displaywidowpenalty=500
	\clubpenalty=500

	\brokenpenalty=500
	\interlinepenalty=500
	
	

	\predisplaypenalty=100
	\postdisplaypenalty=100
	\adjdemerits=250000

	\doublehyphendemerits=250000
	\finalhyphendemerits=250000
	\lineskip=0pt % \baselineskip
	\righthyphenmin=2
	\lefthyphenmin=2
	\uchyph=0

\frenchspacing
\newcommand{\kerning}[2]{\letterspace to #1\naturalwidth{#2}}
\newcounter{nnn}
%%%%%%%%%%%%%%%%%%%%%%%%%%%%%%%%%%%%%%%%%%%%%%%%%%%%%%%%%%%%%%%%%%%%%%%%%%%%%
%	Настройка постоянных констант текста
%%%%%%%%%%%%%%%%%%%%%%%%%%%%%%%%%%%%%%%%%%%%%%%%%%%%%%%%%%%%%%%%%%%%%%%%%%%%%

\newcommand{\theyear}{$2020$}
\newcommand{\theseason}{XLV}
\newcommand{\deadline}{{1~февраля}}

%%%%%%%%%%%%%%%%%%%%%%%%%%%%%%%%%%%%%%%%%%%%%%%%%%%%%%%%%%%%%%%%%%%%%%%%%%%%%
%	Изменение расстояния для списка \itemize
%%%%%%%%%%%%%%%%%%%%%%%%%%%%%%%%%%%%%%%%%%%%%%%%%%%%%%%%%%%%%%%%%%%%%%%%%%%%%
%\newenvironment{itemize}%
%{\begin{list}{$\bullet$}{\setlength{\itemsep}{0cm}%
%\setlength{\parsep}{0cm}\setlength{\topsep}{2mm}\item}}{\end{list}}

%%%%%%%%%%%%%%%%%%%%%%%%%%%%%%%%%%%%%%%%%%%%%%%%%%%%%%%%%%%%%%%%%%%%%%%%%%%%%
%	Специальные команды верстки
%%%%%%%%%%%%%%%%%%%%%%%%%%%%%%%%%%%%%%%%%%%%%%%%%%%%%%%%%%%%%%%%%%%%%%%%%%%%%

\newcommand{\ItemWindowPar}[3]{
\parbox[t]{\textwidth - \leftmargin}{
	\par\vspace{-0.8\baselineskip}\par
	\begin{window}[#1,l,#2,{}]
	#3
	\end{window}\par\vspace{0.3\baselineskip}\par
}
}

\newcommand{\ItemWindowParRight}[3]{
\parbox[t]{\textwidth - \leftmargin}{
	\par\vspace{-0.8\baselineskip}\par
	\begin{window}[#1,r,#2,{}]
	#3
	\end{window}\par\vspace{0.3\baselineskip}\par
}
}

%%%%%%%%%%%%%%%%%%%%%%%%%%%%%%%%%%%%%%%%%%%%%%%%%%%%%%%%%%
%	размеры шрифтов
%%%%%%%%%%%%%%%%%%%%%%%%%%%%%%%%%%%%%%%%%%%%%%%%%%%%%%%%%%

\newlength{\SkipBetweenLine}
\newlength{\SkipBetweenSmallLine}
\setlength{\SkipBetweenLine}{14pt} % 15.
\setlength{\SkipBetweenSmallLine}{12.5pt}
\newfontfamily\afont[Script=Arabic]{KacstOffice}
\spaceskip=5pt plus 2.5pt minus 2.5pt % ограничим растяжимость пробелов между словами

\newcommand{\TableSize}{\fontsize{9.5pt}{\SkipBetweenLine}\selectfont}
\newcommand{\QuoteSize}{\fontsize{11pt}{\SkipBetweenLine}\selectfont}
\newcommand{\IntroSize}{\fontsize{11.2pt}{\SkipBetweenLine}\selectfont}
\newcommand{\EnglishIntroSize}{ \fontsize{11.2pt}{\SkipBetweenLine}\selectfont}
\newcommand{\TasksSize}{\fontsize{10pt}{\SkipBetweenLine}\selectfont}
\newcommand{\FooterSize}{\fontsize{8.3pt}{\SkipBetweenSmallLine}\selectfont}
\newcommand{\HeaderSize}{\fontsize{7.6pt}{\SkipBetweenLine}\selectfont}

%%%%%%%%%%%%%%%%%%%%%%%%%%%%%%%%%%%%%%%%%%%%%%%%%%%%%%%%%%%%%%%%%%%%%%%%%%%%%
%	Установка колонтитулов
%%%%%%%%%%%%%%%%%%%%%%%%%%%%%%%%%%%%%%%%%%%%%%%%%%%%%%%%%%%%%%%%%%%%%%%%%%%%%

\newcommand{\MarginOutlet}{0.5cm} % допуск выхода на поля логотипов и колонтитулов
\newcommand{\RuleThickness}{0.7pt}
\newcommand{\HeaderRuleSkip}{-2pt}

\newcommand{\Header}{\FooterSize%
   \scshape К{\HeaderSize расноярская}%
\hspace{4pt}Л{\HeaderSize етняя}%
\hspace{4pt}Ш{\HeaderSize кола}%
\hspace{4pt}—%
\hspace{4pt}\theyear}

\newlength{\HeaderLength}
\settowidth{\HeaderLength}{%\fontencoding{T2A}\selectfont
\Header}

\newcommand{\LeftHeader}{\parbox{0pt}{\mbox{\Header}}\rule[\HeaderRuleSkip]{\HeaderLength}{\RuleThickness}}
\newcommand{\RightHeader}{
	\parbox{0pt}{\parbox{\HeaderLength}{\hfill\mbox{\Header}}}%
	\mbox{}{\makebox[\HeaderLength]{\hfill\rule[\HeaderRuleSkip]{\HeaderLength}{\RuleThickness}}}}

\pagestyle{myheadings}
\makeatletter

\renewcommand{\@oddhead}{\hspace{-\MarginOutlet}\LeftHeader}
\renewcommand{\@evenhead}{\parbox{\textwidth + \MarginOutlet}{\hfill\RightHeader}}
\renewcommand{\@oddfoot}{{\sloppy\hfill\FooterSize\thepage}}
\renewcommand{\@evenfoot}{{\sloppy\FooterSize\thepage\hfill}}

\makeatother

%%%%%%%%%%%%%%%%%%%%%%%%%%%%%%%%%%%%%%%%%%%%%%%%%%%%%%%%%%%%%%%%%%%%%%%%%%%%%
\newcommand{\RimNum}[1]{\uppercase\expandafter{\romannumeral #1\relax}}
\begin{document}
%%%%%%%%%%%%%%%%%%%%%%%%%%%%%%%%%%%%%%%%%%%%%%%%%%%%%%%%%%
%% 			Первая страница
%%%%%%%%%%%%%%%%%%%%%%%%%%%%%%%%%%%%%%%%%%%%%%%%%%%%%%%%%%

\thispagestyle{empty}
\rule{0pt}{0pt}
\enlargethispage{1\baselineskip}

{\centering
\mbox{}
\vspace{-4\baselineskip}

КРОО „Красноярская Летняя Школа“\\
Министерство образования Красноярского края\\
Сибирский федеральный университет\\
Красноярская университетская гимназия \textnumero $1$ – Универс

\vfill
\vspace{-4\baselineskip}
\includegraphics[scale=1]{./images/klsh_logo.pdf}\par
\vspace{-2\baselineskip}
{\fontsize{20}{60}\selectfont Вступительное задание}
\vfill
\ \\
\theyear{}

}

%%%%%%%%%%%%%%%%%%%%%%%%%%%%%%%%%%%%%%%%%%%%%%%%%%%%%%%%%%
%% 			Дорогой друг
%%%%%%%%%%%%%%%%%%%%%%%%%%%%%%%%%%%%%%%%%%%%%%%%%%%%%%%%%%

%\begin{comment}

\newpage

\newlength{\LogoHeight}
\setlength{\LogoHeight}{67pt}

\vspace{1.4\baselineskip}

\input dear-utf8.tex

%\end{comment}

\TasksSize

%%%%%%%%%%%%%%%%%%%%%%%%%%%%%%%%%%%%%%%%%%%%%%%%%%%%%%%%%%%%%%%%%%%%%%%%%%%%%
%	Отступы в списках пакета paralist:
%%%%%%%%%%%%%%%%%%%%%%%%%%%%%%%%%%%%%%%%%%%%%%%%%%%%%%%%%%%%%%%%%%%%%%%%%%%%%

\setdefaultleftmargin{21pt}{2.5em}{2.5em}{}{}{}
\plparsep = 0.5\baselineskip

%%%%%%%%%%%%%%%%%%%%%%%%%%%%%%%%%%%%%%%%%%%%%%%%%%%%%%%%%%
%% 			Точные науки
%%%%%%%%%%%%%%%%%%%%%%%%%%%%%%%%%%%%%%%%%%%%%%%%%%%%%%%%%%

\newpage
\noindent{\huge\scshape Направление Точных Наук\\(9 задач)}

\vspace{2\baselineskip}

\input exact-utf8_phisics.tex

\vspace{2\baselineskip}

% \newpage
\input exact-utf8_math.tex

\vspace{2\baselineskip}

% \newpage
\input exact-utf8_it.tex

%%%%%%%%%%%%%%%%%%%%%%%%%%%%%%%%%%%%%%%%%%%%%%%%%%%%%%%%%%
%% 			Естественные науки
%%%%%%%%%%%%%%%%%%%%%%%%%%%%%%%%%%%%%%%%%%%%%%%%%%%%%%%%%%

\newpage
\noindent{\huge\scshape Направление Естественных Наук\\(9 задач)}

\vspace{2\baselineskip}

\input natural-utf8_chemistry.tex

\vspace{2\baselineskip}

%\newpage
\input natural-utf8_medicine.tex

\vspace{1\baselineskip}

% \newpage
\input natural-utf8_biology.tex

%%%%%%%%%%%%%%%%%%%%%%%%%%%%%%%%%%%%%%%%%%%%%%%%%%%%%%%%%%
%% 			Общественные науки
%%%%%%%%%%%%%%%%%%%%%%%%%%%%%%%%%%%%%%%%%%%%%%%%%%%%%%%%%%

\newpage
\noindent{\huge\scshape Направление Общественных Наук\\(9 задач)}


\vspace{1\baselineskip}

\rule{0pt}{0pt}\hfill\parbox[t]{0.923\textwidth}{
{\it\small
\parskip=6pt\parindent=0pt
\noindent Дорогой друг!%\newline %\parfillskip=1fil

Если ты хочешь поступить на направление общественных наук, реши для этого любые две задачи по математике в дополнение к задачам направления.
}
}

\vspace{2\baselineskip}

\input society-utf8_economy.tex

\vspace{2\baselineskip}

% \newpage
\input society-utf8_history.tex

\vspace{2\baselineskip}

%\newpage
\input society-utf8_society.tex

%%%%%%%%%%%%%%%%%%%%%%%%%%%%%%%%%%%%%%%%%%%%%%%%%%%%%%%%%%
%% 			Филологические науки
%%%%%%%%%%%%%%%%%%%%%%%%%%%%%%%%%%%%%%%%%%%%%%%%%%%%%%%%%%

\newpage
\noindent{\huge\scshape Направление Филологических \mbox{Наук}\\(9 задач)}

\vspace{2\baselineskip}

\input lingua-utf8_culture.tex

\vspace{2\baselineskip}

% \newpage
\input lingua-utf8_lingua.tex

\vspace{2\baselineskip}

% \newpage
\input lingua-utf8_literature.tex

%%%%%%%%%%%%%%%%%%%%%%%%%%%%%%%%%%%%%%%%%%%%%%%%%%%%%%%%%%
%% 			Прощальный спам
%%%%%%%%%%%%%%%%%%%%%%%%%%%%%%%%%%%%%%%%%%%%%%%%%%%%%%%%%%

%\begin{comment}

\newpage

\setlength{\LogoHeight}{67pt}

\vspace{1.4\baselineskip}

\input goodbye-utf8.tex
\newpage

%\end{comment}

%%%%%%%%%%%%%%%%%%%%%%%%%%%%%%%%%%%%%%%%%%%%%%%%%%%%%%%%%%
%% 			Канеза!
%%%%%%%%%%%%%%%%%%%%%%%%%%%%%%%%%%%%%%%%%%%%%%%%%%%%%%%%%%

\begin{print}

\spaceskip=3.55pt plus 3.5pt minus 3.5pt

\enlargethispage{\baselineskip}

\pagestyle{myheadings}
\makeatletter

\renewcommand{\@oddfoot}{}
\renewcommand{\@evenfoot}{}

\makeatother
\thispagestyle{empty}

 
%  \begin{center}
%    {\tiny\null\vfill
%           Напечатано при поддержке КГАУ ``КРИТБИ'' \\
%Красноярский инновационно-технологический бизнес-инкубатор \\
%www.kritbi.ru, www.facebook.com/kritbi/, https://vk.com/kritbi{\_}krsk\\
%Уч.-изд. л. $1$,$5$. Печать офсетная\\
%тираж $2000$ экз.\\
%Отпечатано в типографии ``Ситалл'', $660074$, Красноярск, ул.~Борисова $14$.\\
%$ $\\
%Запишись на бесплатные занятия  в ближайший\\
%Центр инновационного молодежного творчества (ЦМИТ)\\
%\vspace{-1.3\baselineskip}
%по тел. ($391$) $201$-$77$-$77$, доб.$2007$}
%  \end{center}
\end{print}

\end{document}

