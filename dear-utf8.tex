\IntroSize

%\vspace{-\baselineskip}

Дорогой друг!\\

\vspace{\baselineskip}
%\vspace{0.5\baselineskip minus 2pt} %plus 0.5pt

В июле-августе \theyear{} года состоится $\theseason{}$ Красноярская Летняя Школа по естественным и гуманитарным наукам (КЛШ). 

Красноярская Летняя Школа~— первое в крае заведение дополнительного образования, известность которого давно перешагнула границы и края, и России. Начиная с $1976$~года КЛШ каждое лето собирает школьников, интересующихся разными областями науки. С ними проводят занятия научные сотрудники Сибирского федерального университета, институтов Российской академии наук, сотрудники университетов и исследовательских лабораторий США и Европы, а также студенты и аспиранты Сибирского федерального, Красноярского государственного медицинского, Московского, Новосибирского, Санкт-Петербургского университетов, Московского физико-технического института, Высшей школы экономики и других ведущих вузов России. Многие из нынешних сотрудников Летней Школы раньше были её школьниками.

В \theyear{} году в Красноярской Летней Школе будут работать четыре учебных направления:
\begin{list}{$\bullet$}{\leftmargin=6mm \labelwidth=5mm \topsep=0mm \labelsep=2mm \itemsep=0pt \parsep=0mm \itemindent=-1pt}
\item направление точных наук (физика,математика, информатика),
\item направление естественных наук (химия, биомедицина , биология),
\item направление общественных наук (экономика, история, право),
\item направление филологических наук (лингвистика, литературоведение).
\end{list}
Ты сможешь самостоятельно выбирать курсы из предложенной учебной программы.
\iffalse У тебя будет возможность  самостоятельно выбрать из предложенной учебной программы курсы, которые будешь посещать. \fi На занятиях ты познакомишься как с традиционными научными взглядами в той или иной научной области, так и с самыми современными достижениями и гипотезами, которые интересуют учёных. Во второй половине дня проводятся интеллектуальные турниры, межнаправленческие практические мини-курсы,   творческие и спортивные студии, где можно подискутировать с лекторами об искусстве и жизни за кружкой чая.

Школа будет проходить в летнем лагере \textsl{Орбита} (ориентировочно, с~$14$ июля по~$3$ августа), расположенном в живописном и экологически чистом месте в окрестностях г.\,Железногорска. В работе школы могут принять участие школьники, оканчивающие восьмой, девятый или десятый классы общеобразовательных школ. Зачисленные школьники частично или полностью оплачивают стоимость пребывания в лагере непосредственно перед началом Школы во время регистрации. Точная сумма взноса, продолжительность и место проведения Школы будут зависеть от объёма финансирования и станут известны к~1~июня \theyear{} года. Ориентировочная величина взноса составит 17\,000 рублей.

\section*{Как поступить в КЛШ}
Школьники зачисляются в КЛШ по результатам конкурсного отбора на одно из научных направлений. В течение года есть несколько возможностей набрать баллы для зачисления в КЛШ:
\begin{compactenum}
% \setlength\itemsep{-0.75em}
	\item[–] участие в Открытой краевой олимпиаде КЛШ (декабрь $2019$\,г.);
	\item[–] участие в Собеседовании КЛШ (март \theyear{}\,г.);
% 	\item[–] участие в Интенсивных школах КЛШ по предметным направлениям;
	\item[–] решение заочного вступительного задания (до $1$ марта \theyear{}\,г.).
\end{compactenum}

Ты можешь участвовать в конкурсе нескольких направлений одновременно, но зачисление происходит на то направление, где ты набрал наибольшее количество баллов по сумме предметов за одно отборочное мероприятие. Обрати внимание, что баллы за участие в различных мероприятиях не складываются.

Приглашение вне конкурса получают только победители и призеры краевых этапов Всероссийской олимпиады школьников (по соответствующим предметам), победители Открытой Зимней олимпиады КЛШ и победители Комплексного научного турнира, проводимого в выездных интенсивах Красноярской Летней Школой.

Участникам очных отборочных мероприятий КЛШ, победителям олимпиад и всем желающим мы рекомендуем решить также заочное вступительное задание.

\section*{Решение вступительного задания}
Для успешного выполнения вступительного задания необходимо решить как можно больше задач по выбранному направлению. Так, в рамках Направления точных наук представлены математика, физика и информатика. Решив задачи только по информатике, ты, скорее всего, не сможешь набрать нужного количества баллов для попадания в Школу. Чем полнее и интереснее будет твоё решение, тем выше станут твои шансы попасть в КЛШ.

При решении вступительного задания можно пользоваться любой помощью, однако в начале или в конце решения каждой задачи нужно сообщить, кто и каким образом тебе помогал, например: \textit{Я решил задачу самостоятельно}, или \textit{Папа подсказал мне, как начать, а дальше я решил сам}, или \textit{Учитель объяснил мне решение, я всё понял и написал сам} и т.\,д. В случае появления в работах нескольких конкурсантов идентичных решений баллы за выполнение соответствующих задач не будут начислены никому.

Решения вступительных заданий принимаются до $1$ февраля \theyear{}~г. по адресу электронной почты \textit{welcome@klsh.ru}.

\textcolor{red}{Лёша Смолянинов сделает текст, как надо оформить вступ.}
%
%\textcolor{red}{IN PROGRESS. В идеале нам надо сделать онлайн формочку, в которую можно будет загрузить задачи и вписать информацию о себе. Либо придумать офлайн шаблон, который надо заполнить, который мы потом разпарсим.}
%
% Оформлять решения лучше в электронном виде (Word, PDF), но можно выполнить и в тетради или на листах бумаги (которые нужно отсканировать/сфотографировать; убедись, что весь текст читаемый). Каждую задачу необходимо сохранить отдельным (и единственным!) файлом. Если файлов несколько, то их необходимо прислать единым архивом. Название каждого файла должно содержать только номер задачи и расширение, например \textit{1.pdf} или \textit{34.rar}.
 
% Следующим шагом заполни анкету в отдельном файле:
% \begin{compactenum}
% % \setlength\itemsep{-0.75em}
% 	\item[$\cdot$] имя, отчество, фамилия;
% % 	\item[$\cdot$] серия и номер паспорта или свидетельства о рождении, кем и когда выдан документ;
% 	\item[$\cdot$] полное название и номер школы, в которой ты учишься;
% 	\item[$\cdot$] класс;
% % 	\item[$\cdot$] домашний адрес вместе с почтовым индексом;
% 	\item[$\cdot$] город проживания;
% 	\item[$\cdot$] телефон, если есть;
% 	\item[$\cdot$] адрес электронной почты;
% 	\item[$\cdot$] дата рождения;
% % 	\item[$\cdot$] имя, отчество и фамилия мамы, её номер телефона;
% % 	\item[$\cdot$] место работы мамы, должность и адрес организации, её рабочий телефон (если есть);
% % 	\item[$\cdot$] имя, отчество и фамилия папы, его номер телефона;
% % 	\item[$\cdot$] место работы папы, должность и адрес организации, его рабочий телефон (если есть);
% 	\item[$\cdot$] направления КЛШ, во вступительном конкурсе которых ты участвуешь, в порядке предпочтения;
% 	\item[$\cdot$] расскажи, пожалуйста, о своих научных интересах, достижениях, увлечениях и т.\,д. Напиши, почему ты хочешь поехать в КЛШ и чего ждёшь от Летней Школы, а также откуда ты узнал о КЛШ. 
% \end{compactenum}

% Анкету можно заполнить в произвольной форме.

% Прикрепи все файлы к письму, в поле «Тема сообщения» укажи «Фамилия\_Имя\_Отчество\_НТН». Если решал задачи нескольких направлений перечисли их через знак подчеркивания, например, «НЕН\_НОН\_НТН».

Решения вступительных заданий принимаются до 1 марта 2020 г. по адресу электронной почты \textit{welcome\textup{@}klsh.ru}. Оформлять решения лучше в электронном виде (Word, PDF), но можно выполнить и в тетради или на листах бумаги (которые нужно отсканировать/сфотографировать; убедись, что весь текст читае- мый). Каждую задачу необходимо сохранить отдельным (и единственным!) файлом. Если файлов несколько, то их необходимо прислать единым архивом. Название каждого фаила — фамилия\_номер задачи и расширение, например Садовский\_1.pdf или Абанов\_34.rar. 

Обрати внимание, что отправляя нам анкету и вступительное задание, ты соглашаешься на обработку и хранение персональных данных {без использования средств автоматизации} в целях организации вступительных испытаний в Красноярскую Летнюю Школу. 
% \textcolor{red}{Никита Кокухин говорит, что фразу ''без использования средств автоматизации'' надо убрать} \textcolor{blue}{Давайте пока воздержимся от убирания, а вопрос формулировки проработаем.}

% Открытая городская олимпиада КЛШ
% --------------------------------
\section*{Открытая зимняя олимпиада КЛШ}
В декабре~$2019$ года для всех желающих школьников 8\,–\,10 классов пройдёт Открытая зимняя олимпиада КЛШ. Она существенно отличается от большинства подобных: её участники решают задачи не по какому-то конкретному предмету, а сразу по всем 24 преметам, представленным в направлениях, работающих в КЛШ. Победителем будет тот, кто одинаково хорошо решает задачи самых разных предметных областей.
Результаты будут подводиться как в общем зачёте, так и отдельно по направлениям. Поскольку тебе будет необходимо перемещаться между этапами по направлениям (и скорость перемещения здесь тоже является немаловажным фактором), советуем надеть удобную одежду и, самое главное,~— комфортную обувь (девушкам мы категорически не рекомендуем туфли на каблуках). Пожалуйста, уточни место, время и дату проведения олимпиады на сайте КЛШ или в группе КЛШ во ВКонтакте. Трое победителей в общем зачёте будут зачислены на любое направление вне конкурса. Победители по направлениям также будут зачислены в КЛШ. 

% \newpage
% Собеседование
% -------------
\section*{Собеседование}
В марте~\theyear{} года для всех школьников, обучающихся в 8\,–\,10 классах и желающих участвовать в работе КЛШ, состоится собеседование. Точная дата и время собеседования станут известны в начале февраля~— подробная информация об этом будет опубликована на сайте и в группе КЛШ во ВКонтакте. Обращаем твоё внимание на то, что собеседование \textbf{не} является обязательным мероприятием наборной кампании, однако участвуя в нём, ты сможешь улучшить свои шансы на поступление. 

Мы рекомендуем тебе участвовать (по возможности) во всех отборочных мероприятиях!

% Контактная информация
% ---------------------

% Телефоны Дирекции КЛШ: +7 (983) 154-72-54, +7 (902) 990-45-97‬
% Адрес электронной почты: klsh@klsh.ru
% Адрес для вступительных заданий: welcome@klsh.ru
% Страницы КЛШ в Интернете: klsh.ru, vk.com/klsh_ru

% Если ты хочешь получать от нас сообщения с оперативной информацией о мероприятиях КЛШ через email-рассылку, заполни форму со своей контактной информацией по адресу https://clck.ru/EJ4z5

% До встречи в КЛШ–\theyear{}!
% \newpage
\section*{Контактная информация}

\newlength{\address}
\settowidth{\address}{Страницы КЛШ в Интернете:\ \ }

\noindent\parbox[t][0pt]{\address} %0.5\textwidth}
{
\noindent Телефон Дирекции КЛШ:\\
\noindent Адрес электронной почты:\\
\noindent Страницы КЛШ в Интернете:\\
}
\parbox[t][0pt]{0.5\textwidth}
{
+7(983)154-72-54, +7(902)990-45-97\\
{\EnglishIntroSize \textit{klsh@klsh.ru}}\\
{\EnglishIntroSize \textit{https://klsh.ru}, 	\textit{https://vk.com/klsh{\_}ru}}
}
\vspace{\baselineskip}
\\ \\ \\
% Если ты хочешь получать от нас сообщения с оперативной информацией о мероприятиях КЛШ через email-рассылку, заполни форму со своей контактной информацией по адресу {\EnglishIntroSize https://clck.ru/EJ4z5}
% \\ \\
До встречи в КЛШ–\theyear{}!
