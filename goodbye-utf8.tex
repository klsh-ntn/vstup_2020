% \section*{Инженерная школа гимназии «Универс»}

% Начиная с 8-го класса, в гимназии «Универс» можно поступить в инженерную школу с углубленным изучением физики, математики и информатики. 

% Кроме того, инженерная школа отличается от остальных наличием дополнительных нетрадиционных видов деятельности:
% \begin{itemize}
%     \item спецкурсы по электронике, робототехнике и 3d-моделированию;
%     \item выездные полипредметные образовательные интенсивы;
%     \item инженерные проекты и дипломные работы в лабораториях гимназии, университетов и предприятий;
%     \item олимпиадная подготовка.
% \end{itemize}

% С 10-го класса идет распределение на подгруппы по информатике и физике. На информатику отводится 7 часов в неделю, где происходит углубленное изучение языков программирования. 

% Наши ученики успешно выступают на различных интеллектуальных соревнованиях, JuniorSkills, предметных и инженерных олимпиадах, что помогает им в поступлении в ведущие вузы страны. 

% Обучение бесплатно. 
% Более подробную информацию можно получить на сайте инженерной школы: \\
% {\EnglishIntroSize http://ishunivers.su/}
% \newpage
% \section*{Центр «Гравитация»}

% В Красноярске проект «Гравитация» многим известен своей научно-развлека\-тельной деятельностью. Здесь можно посетить научное шоу, зеркальный лабиринт, отметить день рождения и весело провести время с одноклассниками и друзьями.  

% Несколько лет назад центр переехал из ТЦ «Комсомолл» в новый офис в Северном, что способствовало изменению направления работы. 

% В 2018-м году «Гравитация»~— это группа интересных проектов в сфере обучения:
% \begin{itemize}
%     \item подготовка к ЕГЭ/ОГЭ;
%     \item подготовка школьников к олимпиадам;
%     \item детские лагери дневного пребывания для начальной школы;
%     \item курсы английского языка для любого возраста.
% \end{itemize}

% Сейчас основная деятельность центра~— подготовка учеников старших классов к ЕГЭ и ОГЭ. Каждый год молодые талантливые преподаватели центра, в прошлом призеры и победители различных олимпиад, добиваются высоких результатов своих учеников и разрабатывают интересную, а главное, эффективную программу обучения. 

% В октябре открывается новый офис в Студгородке, в котором снова появится возможность проведения детских дней рождений.

% Более подробную информацию можно получить на сайте: \\
% {\EnglishIntroSize       http://gravity24.ru/}