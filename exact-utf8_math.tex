\noindent{\Huge\scshape М}{\LARGE\scshape атематика}\\
\rule[0.5\baselineskip]{\textwidth}{1pt}

\vspace{0\baselineskip}

\subsection*{Задание 4.}

В очереди на посадку в $100$-местный самолёт стоит $100$~человек. Первый человек в очереди потерял свой посадочный талон и сел на случайное место из свободных. Все следующие за ним люди выбирали своё место (указанное в талоне), если оно было свободно, иначе выбирали случайное незанятое место. С какой вероятностью последний человек в очереди займёт своё место, указанное в талоне?

\subsection*{Задание 5.}

Дан произвольный треугольник $ABC$. Двумя разрезами раздели его на части так, чтобы из них можно было собрать равнобедренный треугольник (равновеликий данному), причём б{\'{о}}льшие стороны этих треугольников должны быть равны.

\subsection*{Задание 6.}

Реши систему уравнений
\begin{equation*}\left\{
\begin{aligned}
    x_2 &= \frac{1}{x_1 + 1},\\
    x_3 &= \frac{1}{x_2 + 1},\\
    x_4 &= \frac{1}{x_3 + 1},\\
    &\ldots\\
    x_{2020} &= \frac{1}{x_{2019} + 1},\\
    x_1 &= \frac{1}{x_{2020} + 1}.
\end{aligned}\right.
\end{equation*}