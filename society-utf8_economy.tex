\noindent{\Huge\scshape Э}{\LARGE\scshape кономика}\\
\rule[0.5\baselineskip]{\textwidth}{1pt}

\vspace{0\baselineskip}

\rule{0pt}{0pt}\hfill\parbox[t]{0.923\textwidth}{
{\it\small
\parskip=6pt\parindent=0pt
% \noindent Дорогой друг!%\newline %\parfillskip=1fil

Для решений первой задачи рекомендуем прочитать главы 2-4 из учебника
Вэриана~Х.~Р. «Микроэкономика. Промежуточный уровень. Современный подход», а для решения третьей — тему 2 из учебника Матвеевой~Т.~Ю. «Введение в макроэкономику».
}
}
\vspace{1\baselineskip}


% \begin{compactenum}
% \setlength\itemsep{-0.25em}
    % \item[19.] 
\subsection*{Задание 19.}
В теории потребительского поведения фигурирует такое понятие, как функция полезности. Вот какое определение дает Хэл Рональд Вэриан: 

\begin{quote}
функция полезности — это такой способ приписывания каждому возможному потребительскому набору некоего численного значения, при котором более предпочитаемым наборам приписываются б\'{о}льшие численные значения, чем менее предпочитаемым.
\end{quote}

Рассмотрим случай, когда наш потребительский набор состоит лишь из двух товаров: пюре и котлет. Функция полезности для некоего агента имеет вид $u(x,y)=\min\{x, y\}$, где $x$ — количество съеденных тарелок картофельного пюре, а $y$ — количество мясных котлет.
\begin{list}{\asbuk{nnn})}{\usecounter{nnn}\leftmargin=6mm \labelwidth=5mm \topsep=0mm \labelsep=2mm \itemsep=0pt \parsep=0mm \itemindent=-1pt}
\item Изобрази в одной плоскости несколько кривых безразличия. Кривые безразличия показывают все наборы $(x,y)$, при которых агент получает одинаковую полезность, то есть ему безразлично, какой из данных наборов выбирать.
\item Теперь рассмотрим бюджетное ограничение нашего агента: пусть нам известно, что цена тарелки пюре равна 20 рублей, а цена котлеты равна 30 рублей. Доход агента составляет 200 рублей. Какую максимальную полезность он может получить?
\item Дирекция решила улучшить жизнь этому агенту и предложила ему альтернативу в получении льгот: либо дирекция понижает цену на пюре в два раза, либо повышает доход в полтора раза. Какую льготу выберет агент, если он максимизирует свою полезность?
\item Может ли агент с такой функцией полезности существовать? Если да, то опиши предпочтения агента относительно пюре и котлет, если нет, то докажи.
\end{list}

% \item[20.] 
\subsection*{Задание 20.}
Фирма OPR строит завод-автомат по производству 3-D принтеров. Срок строительства — 3 года. Для работы на стройке требуются работники, имеющие соответствующую квалификацию. Эту квалификацию можно получить только в учебном центре фирмы OPR. Фирма принимает на работу граждан, имеющих соответствующую квалификацию. Информация от учебного центра представлена в таблице\medskip

\begin{tabular}{lccc}
\toprule
Квалификация & 
\begin{tabular}{@{}c@{}} Срок \\ обучения \\ \small{(месяцев)}\end{tabular} &
\begin{tabular}{@{}c@{}} Оплата \\ обучения \\ \small{(тыс. руб.} \\ \small{за 6 месяцев)} \end{tabular} &
\begin{tabular}{@{}c@{}} Доход \\ работника \\  \small{(тыс. руб.} \\ \small{в месяц)} \end{tabular} \\
\midrule
Монтажник & 6 & 50 & 70 \\
Специалист & 12 & 70 & 120 \\
Администратор & 18 & 90 & 240 \\
\bottomrule
\end{tabular}\medskip

\noindent Оплата за обучение не включает стоимость учебных материалов (1 тыс. руб. в месяц для всех специальностей) и расходы на питание (1 тыс. руб. в месяц). Обучение проводится только с отрывом от работы.

Проучившись 6 месяцев на курсах для специалистов,
Артём стал сомневаться в правильности своего выбора и решил пересмотреть свое решение.

Расходы, понесенные Артёмом за 6 месяцев обучения, ему возмещены не будут. Артём может поступить на работу, требующую квалификации монтажника, без дополительного обучения.
Артём может продолжить обучение как специальности, так и на курсах администраторов (в последнем случае — оплатив разницу в плате за обучение). Артём хочет получить максимальный суммарный располагаемый доход за оставшиеся время строительства завода. Располагаемый доход рассматривается как разница между полученным доходом и понесенными затратами; при этом работа в фирме приносит больший доход по сравнению с другими возможными вариантами трудоустройства. Варианты своего дальнейшего трудоустройства и факторы, с ним связанные, Артём пока не рассматривает. Инфляция в расчет не принимается.

\begin{list}{\asbuk{nnn})}{\usecounter{nnn}\leftmargin=6mm \labelwidth=5mm \topsep=0mm \labelsep=2mm \itemsep=3pt \parsep=1.56mm \itemindent=-1pt}
\item Как следует поступить Артёму?
\item Как должно измениться решение Артёма в ситуации, когда существует подоходный налог?
% Шкала налогов представлена в таблице:

\begin{tabular}{lr}
\toprule
Диапазон дохода & Налог \\
\midrule
С годового дохода до 1 млн. руб. & 12\% \\
С годового дохода свыше 1 млн. руб. & 20\% \\
\bottomrule
\end{tabular}

Налог будет взиматься только с доходов, которые Артём получит, поступив на работу.

\item При каком уровне инфляции решения, полученные в предыдущих пунктах, не изменятся?
\end{list}

%\newpage    

% \item[21.] 
\subsection*{Задание 21.}
Экономика государства характеризуется следующими показателям

%\begin{tabular}{lr}
%\toprule
\begin{tabbing}
Отчисления за возмещение потребленного капитала \qquad \=\qquad \= Сумма, \small{(млрд. руб.)}    \kill
\textbf{Макроэкономический показатель} \> \hspace{-25pt}\textbf{Сумма}, {\small(млрд. руб.)}\\
%\begin{tabular}{@{}c@{}} Сумма \\  \small{(млрд. руб.)} \\ \end{tabular}\\
%\midrule
Заработная плата \> 627 \\
Проценты по государственным облигациям \> 25 \\
Трансфертные платежи \> 39 \\
Доходы, полученные за рубежом \> 16 \\
Отчисления за возмещение потребленного капитала \> 87 \\
Арендная плата \> 14 \\
Косвенные налоги на бизнес \> 18 \\
Дивиденды \> 15 \\
Нераспределенная прибыль корпораций \> 16 \\
Экспорт \> 153 \\
Процентные платежи \> 30 \\
Взносы на социальное страхование \> 17 \\
Доходы, полученные иностранцами \> 28 \\
Государственные закупки товаров и услуг \> 145 \\
Прибыль корпораций \> 48 \\
Индивидуальные налоги \> 55 \\
Инвестиции в производственное оборудование \> 91 \\
Инвестиции в строительство \> 143 \\
Прирост товарно-материальных запасов \> 20 \\
Потребительские расходы домашних хозяйств \> 585 \\
Импорт \> 220 
\end{tabbing}
%\bottomrule
%\end{tabular}
%\bigskip

\begin{list}{\asbuk{nnn})}{\usecounter{nnn}\leftmargin=6mm \labelwidth=5mm \topsep=0mm \labelsep=2mm \itemsep=0.3pt \parsep=0.06mm \itemindent=-1pt}
\item Чему равна величина чистого экспорта?
\item Чему равен ВВП (валовый внутренний продукт)? 
\item Чему равен ВНП (валовый национальный продукт)?
\item Чему равен личный доход?
\end{list}    