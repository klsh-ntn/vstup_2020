\noindent{\Huge\scshape Ф}{\LARGE\scshape изика}\\
\rule[0.5\baselineskip]{\textwidth}{1pt}

\vspace{0\baselineskip}

\subsection*{Задание 1.}

В бассейне плавает лодка водоизмещением $3$~тонны. В дне лодки открывают клапан и в неё поступает некоторое количество воды, затем клапан закрывают, при этом лодка продолжает оставаться на плаву. Затем в лодку кладут бревно массой $300$~килограмм и оно в ней плавает, не касаясь лодки. При этом выяснилось, что уровень воды в лодке после того, как туда положили бревно, и уровень воды в бассейне совпадают. Найди массу влитой в лодку воды.

\subsection*{Задание 2.}

Хипстер Валерий на гироскутере с колесом диаметром $20$~сантиметров подъехал к высокому бордюру высотой $5$~сантиметров. Какую минимальную силу должно развить колесо гироскутера для преодоления данного препятствия при условии, что вес Валерия вместе с гироскутером равен $50$~килограммам? Силой трения пренебречь. Найди максимальное отношение радиуса колеса к высоте бордюра, если мощность колеса~$1$~КВт.

\subsection*{Задание 3.}
Известно, что айсберги, оказавшиеся в тёплых водах, могут переворачиваться. В этой задаче предлагается исследовать это явление, воспользовавшись следующей моделью. Считай, что изначально айсберг имеет форму цилиндра высоты $L$ и диаметра $D$, таяние происходит только в воде и уменьшает только диаметр айсберга. Пренебреги изменением высоты айсберга, а также считай, что часть айсберга, исходно находившаяся над водой, под воду не опускается. Таким образом, в любой момент времени наш «модельный» айсберг представляет собой два соосных цилиндра, один из которых всегда находится над водой.

\begin{enumerate}
\item В исходный момент времени рассмотри моменты сил тяжести и Архимеда, действующих на айсберг. Пусть айсберг по какой-то причине (например, из-за волнения на воде) отклонился от вертикали на малый угол $\varphi$. Запиши моменты сил относительно центра айсберга $O$. Покажи, что результирующий момент сил будет возвращать айсберг в положение равновесия.
\item Пусть диаметр подводной части уменьшился и стал $(1 - x) D$. Объясни качественно, что происходит с моментами сил тяжести и Архимеда относительно точки $O$: как изменяются плечи сил и сами силы?
\item Считая, что $x \ll 1$, запиши выражения для моментов сил тяжести и Архимеда относительно точки $O$. Найди $x$, при котором айсберг переходит в положение неустойчивого равновесия.
\end{enumerate}   
