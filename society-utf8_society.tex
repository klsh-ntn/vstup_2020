\noindent{\Huge\scshape П}{\LARGE\scshape раво}\\
\rule[0.5\baselineskip]{\textwidth}{1pt}

\vspace{0\baselineskip}

\subsection*{Задание 25.}
    В гражданском праве РФ существует принцип «свободы договора». Согласно статье 421 Гражданского кодекса РФ \textit{граждане и юридические лица свободны в заключении договора}.
    \\
    Согласно п. 2 ст. 16 Закон РФ от 07.02.1992 N 2300-1 «О защите прав потребителей» \textit{запрещается обусловливать приобретение одних товаров (работ, услуг) обязательным приобретением иных товаров (работ, услуг)}.
    \\
    При этом \textit{все владельцы транспортных средств}, согласно Федеральному закону от 25.04.2002 N 40-ФЗ «Об обязательном страховании гражданской ответственности владельцев транспортных средств», \textit{обязаны страховать риск своей гражданской ответственности, которая может наступить вследствие причинения вреда жизни, здоровью или имуществу других лиц при использовании транспортных средств}. То есть, обязаны заключить договор со страховой компанией.
    \\
    \textit{Не видишь ли ты противоречий в представленных тебе нормах? Не противоречит ли в таком случае обязанность по страхованию гражданской ответственности владельцев транспортных средств основам гражданского права? Ответ поясни со ссылкой на российское законодательство}. 

\subsection*{Задание 26.}
    Сидоров Миша — непутевый студент последнего курса Юридического факультета выдуманного ВУЗа. Он заказал дипломную работу по объявлению у индивидуального предпринимателя Иванова И.И. После оплаты ему была передана дипломная работа. На защите студент получил оценку «неудовлетворительно», поскольку работа была написана на основе недействующего законодательства, и процент уникальности текста составлял менее 5\%. 
    \\
    \textit{Со ссылкой на российское законодательство ответь на поставленные вопросы:
    \begin{enumerate}
    \item Может ли Миша вернуть денежные средства, уплаченные за заказ дипломной (курсовой) работы у индивидуального предпринимателя Иванова?
    \item Каков порядок возврата денежных средств? Если он откажется вернуть деньги, можно ли требовать их в суде? 
    \item Изменится ли ответ, если дипломная работа была заказана у юридического лица или у физического лица, не имеющего статуса индивидуального предпринимателя?
    \end{enumerate}}
    
\subsection*{Задание 27.}
    Писательница Вера Засулич выстрелом из револьвера тяжело ранила петербургского градоначальника Ф.Ф. Трепова. По делу присяжными заседателями вынесен оправдательный вердикт.
    \\
    Банкир Станислав Кроненберг $15$ минут в полную силу порол связкой прутьев свою $7$-летнюю дочь за кражу чернослива. Был полностью оправдан Санкт-Петербургским окружным судом с участием присяжных заседателей.
    \\
    Какое решение присяжные заседатели приняли в романе Ф.М. Достоевского «Братья Карамазовы» мы тебе не расскажем, прочитай сам(а).
    \\
    В настоящее время дел, рассмотриваемых с привлечением суда присяжных, ничтожно мало. \textbf{Районные суды} за год успели рассмотреть около ста дел с присяжными заседателями, при этом около $30\%$ подсудимых оправданы\footnote{Здесь и далее в абзаце – статистические данные за 2018 год.}. Процент оправдательных вердиктов \textbf{в региональных судах} с присяжными вдвое меньше — примерно $15\%$. Напомним, средний процент оправдательных приговоров в России — $0,25\%$. 
    \\
    \textit{Скорее всего, ты имеешь некоторое представление о присяжных заседателях и их работе.  На основании твоих знаний, истории России, теоретической информации и нормативных источников ответь на предложенные вопросы:
    \begin{enumerate}
    \item Проследи историю развития института присяжных заседателей в России, отмечая ключевые нормативные акты, которые регулировали этот институт (для современной России можешь отметить ключевые акты Верховного суда и Конституционного суда РФ). 
    \item Кратко опиши модель, по которой сейчас действует суд присяжных заседателей в России.
    \item Почему, по твоему мнению, доля оправдательных вердиктов, вынесенных присяжными заседателями, значительно превосходит долю оправдательных приговоров, выносимых профессиональными судями?
    \item В чем, на твой взгляд, назначение института присяжных заседателей? Выполняет ли институт присяжных заседателей это назначение в современной России? Ответ обоснуй.
    \end{enumerate}}