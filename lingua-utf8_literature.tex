\noindent{\Huge\scshape Л}{\LARGE\scshape итературоведение}\\
\rule[0.5\baselineskip]{\textwidth}{1pt}
% \newlength{\versewidth}
\vspace{-1.5\baselineskip}

% \begin{compactenum}
% \setlength\itemsep{-0.25em}
%     \item[1.] 

\subsection*{Задание 34.}

    Как известно, роман в стихах А.С. Пушкина «Евгений Онегин» относится к лироэпическому роду литературы. \\
    Прочитай \RimNum{37} и \RimNum{38} строфы первой главы и \RimNum{43}, \RimNum{44}, \RimNum{45} строфы пятой главы романа и ответь на вопросы: а) почему роман относят к данному роду литературы? б) что в нём есть от лирики, а что от эпоса? \\
    Для аргументации можешь ссылаться и на другие части романа. Для более глубокого понимания особенностей лироэпики советуем перечитать роман полностью.
    
    Лироэпика — четвёртый род литературы, симбиоз лирики и эпики. Такая информация о родах содержится в работе И.Н. Сухих «Структура и смысл: Теория литературы для всех»: «В начале 1840-х годов гегельянскую трактовку, с некоторым упрощением и схематизацией, воспроизвел в русской эстетике и критике В.Г. Белинский, благодаря которому проблема получила чёткую формулировку — „Разделение поэзии на роды и виды“ (заглавие статьи-трактата, 1841). Объективное, внешнее событие, о котором рассказывается, повествуется; субъективное, внутреннее чувство, переживание, которое высказывается, изливается; наконец, внешне-внутреннее действие, представленное в форме непосредственных человеческих столкновений, — таковы, пожалуй, краткие формулы эпического, лирического и драматического родов. Или ещё короче: эпос — повествование о событии, драма — изображение действия, лирика — выражение переживания».

\subsection*{Задание 35.}
    % \item[2.] 
    Прочти рассказ С. Довлатова «Поплиновая рубашка» и ответь на вопросы:
    \begin{compactenum}
        \setlength\itemsep{-0.25em}
        \item[а)] Какой тип композиции использовал писатель? Свой ответ подтверди и проиллюстрируй цитатами из анализируемого рассказа.
        \item[б)] Проанализируй хронотоп\footnote{ Определение хронотопа ты сможешь найти в работе М.М. Бахтина «Формы времени и хронотопа в романе».}: \romannumeral 1) в каких городах происходит действие? \romannumeral 2) в каких годах происходит действие?
        \item[в)] Отличается ли интерпретация текста до и после анализа хронотопа? Если да, то как?
    \end{compactenum}
    
    Для ответа на последний вопрос тебе может помочь выполнение следующего алгоритма: а) прочитать текст в первый (ознакомительный) раз и записать сюжет в хронологической последовательности; б) перечитать текст, обращая внимание на любые детали, уточняющие хронотоп, и записать сюжет в хронологической последовательности с принятием во внимание всех нюансов.

\subsection*{Задание 36.}
    % \item[3.] 
    Прочитай стихотворения Б.К. Лившица и В.Я. Брюсова и проанализируй их:
    \begin{compactenum}
        \setlength\itemsep{-0.25em}
        \item[а)] Какие мотивы\footnote{О том, что понимается под мотивом, ты можешь прочитать в монографии Силантьев И.В. Поэтика мотива. М.: Языки славянской культуры, 2004. С. 86–89.} являются общими для этих текстов?
        \item[б)] Проанализируй, к каким текстам отсылают эти стихотворения и опиши способ отсылки: какую функцию выполняют аллюзии, типы цитирования и образы зейденбергской пыли, чуда, природы, хлебопашцев, рыбаков и Иоанна Богослова?
    \end{compactenum}
    
    \hfill
    \begin{multicols}{2}
    \footnotesize
        \noindent
        Когда, о Боже, дом Тебе построю, \\
        Я сердце соразмерить не смогу\\
        С географическою широтою,\\
        И севером я не пренебрегу. \\
        
        \noindent
        Ведь ничего действительнее чуда\\
        В обычной жизни не было и нет:\\
        Кто может верно предсказать, откуда \\
        Займется небо и придет рассвет? \\
        
        \noindent
        И разве станет всех людских усилий,\\
        Чтоб Царствия небесного один — \\
        Один лишь луч, \\ сквозь зейденбергской пыли, \\
        На оловянный низошел кувшин? \\
        
        \noindent
        Кто хлебопашествует и кто удит \\
        И кто, на лиру возложив персты, \\
        Поет о том, что времени не будет, — \\
        Почем нам знать, откуда идешь Ты? \\
        
        \noindent
        Во всех садах плоды играют соком.  \\ 
        Ко всем Тебе прямы Твои стези: \\ 
        Где ни пройдешь, \\ Ты всё пройдешь востоком – \\
        О, только сердце славою пронзи! 

        \begin{center}
        Б.К. Лившиц, 1919 
        \end{center}
    \columnbreak
        \noindent
        Единый раз свершилось чудо: \\
        Порвалась связь в волнах времен. \\
        Он был меж нами, и отсюда \\
        Смотрел из мира в вечность он. \\
        Все эти лики, эти звери, \\
        И ангелы, и трубы их \\
        В себе вмешали в полной мере \\
        Грядущее судеб земных. \\
        Но в миг, когда он видел бездны, \\
        Ужели ночь была и час, \\
        И все вращался купол звездный, \\
        И солнца свет краснел и гас? \\
        Иль высшей волей провиденья \\
        Он был исторгнут из времен, \\
        И был мгновеннее мгновенья \\
        Всевидящий, всезрящий сон? \\
        Все было годом или мигом, \\
        Что видел, духом обуян, \\
        И что своим доверил книгам \\
        Последний вестник Иоанн? \\
        Мы в мире времени, — отсюда \\
        Мир первых сущностей незрим. \\
        Единый раз свершилось чудо — \\
        И вскрылась вечность перед ним. \\
        
        \begin{center}
        В.Я. Брюсов, «Патмос», 1902
        \end{center}
    \end{multicols}
    

        
    

        

